\chapter*{Введение}\addcontentsline{toc}{chapter}{Введение}
Документ, который вы читаете, представляет собой инновационный конспект по курсу <<Операционные системы>>.
Инновационность заключается в постоянном использовании буквы <<ё>>, где надо и где нет (соответственно, в версии без буквы <<ё>> эта инновационность отсутствует всюду кроме этого введения).

Всё, что не касается технической стороны предметов, затрагиваемых в данном документе, является моей личной точкой зрения, которая не может считаться коль сколь-нибудь авторитетной в вопросах, в которых я не считаю себя экспертом (например, авторское право).
Тем не менее, любые конструктивные замечания и исправления принимаются с радостью.

\section*{Соглашения}
Определения и теоремы выделяются классически, используется сквозная нумерация внутри глав.
Словосочетания, важности которых для данного текста не хватает, чтобы носить гордое название <<Определение>>, выделены \emph{курсивом}, точно также выделяются места с логическим ударением.

\begin{definition}
Научная работа~--- работа в которой есть хотя бы одно доказательство.
\end{definition}

\begin{theorem}
Этот текст~--- научная работа.
\end{theorem}

\begin{proof}
Тут есть доказательство.
\end{proof}

Все приводимые примеры кода по возможности компилируются и оттестированы.
В заголовках листингов указывается имя файла, в котором его нужно разместить, а в комментариях в конце листинга иногда указываются команды, которые необходимо ввести в оболочку, чтобы этот файл скомпилировать (слинковать, запустить, \dots).
Листинги без заголовков представляют собой, или псевдокод, или вывод каких-то программ, или примеры, которые невозможно запустить по каким-то причинам.

По отношению к структурному элементу многих языков программирования, обычно называемому <<функцией>>, (из религиозных соображений) применяется термин <<процедура>> в не зависимости от того, имеется ли у этого элемента возвращаемое значение.

Все факты из данного документа настоятельно рекомендуется сверять с соответствующими мануалами, а все исходники изучать, компилировать и запускать, поскольку они могут содержать как случайные, так и не очень, ошибки.

\section*{Состав}
В каждый момент времени этот документ не окончен, находится в разработке и в бета-версии.
Секции и даже целые главы могут быть пропущены из-за моей лени, недостатка времени или слишком низкого приоритета (например, излагается очень простой материал, съедобную информацию по которому легко найти в сети).

Курс рассчитан на два семестра (а может быть даже и на три).
В первом семестре проводится общее знакомство с системами типа UNIX-like без каких-либо серьёзных подробностей.
Примерно половина второго семестра посвящена трешу POSIX API, а вторая половина~--- каким-то более-менее общим (а потому хоть немного интересным) вещам.
Третий семестр как бы рассчитан на advanced темы и мощных студентов и, если когда-нибудь и будет существовать, то, видимо, в качестве факультатива (семинаров, или чего-то такого).

В принципе, первый семестр~--- абсолютный треш и всё что в нём содержится можно самостоятельно изучить за неделю или две, ещё есть какие-то лабораторные работы, но они тупые, студентам лень их делать, а мне проверять.
В каждую конкретную <<интересную>> тему второго семестра можно закопаться очень глубоко, однако это делать опасно, ибо можно обратно уже не вылезти.
Про третий семестр ничего не знаю.
